\documentclass[12pt,letterpaper]{article}

\newcommand\tab[1][1cm]{\hspace*{#1}}
\usepackage[utf8]{inputenc}
\usepackage[letterpaper]{geometry}
\usepackage{amsmath}
\usepackage{gensymb}
\usepackage{amssymb}
\usepackage{xcolor}
\usepackage{booktabs}
\usepackage{rotating}
\usepackage{fancyhdr}
\usepackage{hyperref}
\usepackage{setspace}
\usepackage{mathtools}
\usepackage{float}
\usepackage{caption}
\usepackage{bm}
\usepackage{enumitem}
\usepackage{pgfplots}
\usepackage{pgfplotstable}
\usepackage[normalem]{ulem}

\usepackage{physics}
\usepackage{tikz}
\usepackage{mathdots}
\usepackage{yhmath}
\usepackage{cancel}
\usepackage{color}
\usepackage{siunitx}
\usepackage{array}
\usepackage{multirow}
\usepackage{tabularx}
\usepackage{extarrows}
\usepackage{scalerel}
\usepackage{graphicx}
\usetikzlibrary{fadings}
\usetikzlibrary{patterns}
\usetikzlibrary{shadows.blur}
\usetikzlibrary{shapes}
\pgfplotsset{compat=newest}
\newcommand{\sfrac}{\genfrac{}{}{}1}

\makeatletter
\newcommand*\bigcdot{\mathpalette\bigcdot@{.5}}
\newcommand*\bigcdot@[2]{\mathbin{\vcenter{\hbox{\scalebox{#2}{$\m@th#1\bullet$}}}}}
\makeatother

\tolerance=1
\emergencystretch=\maxdimen
\hyphenpenalty=10000
\hbadness=10000

\captionsetup[table]{skip=0pt,singlelinecheck=off}
\doublespacing 

\geometry{top=0.6in, bottom=0.7in, left=1.0in, right=1.0in}
\renewcommand{\headrulewidth}{0pt} 
\renewcommand{\footrulewidth}{0pt} 
\setlength\headsep{0.333in}


\title{\textbf{Smart Cart Collisions Exploration}}
\author{\textbf{Siddharth Nema}\\SPH4U0 \\Ms.\hspace{-1mm} Ally}

\begin{document}
\maketitle
\newpage

\section{Collision Observations}
\textbf{The Five Collision Scenarios:}
\vspace{-4mm}
\begin{enumerate}
	\itemsep-0.75em
	\item \textbf{Collision 1:} Both carts equal mass | Bounce of magnets | Cart 2 (Blue) at rest
	\item \textbf{Collision 2:} Both carts equal mass | Bounce of magnets | Roll towards each other
	\item \textbf{Collision 3:} $m_{2} = 2m_{1}$ | Bounce of magnets | Cart 2 (Blue) at rest
	\item \textbf{Collision 4:} $m_{2} = 2m_{1}$ | Bounce of magnets | Roll towards each other
	\item \textbf{Collision 5:} Both carts equal mass | Stick together after | Cart 2 (Blue) at rest
\end{enumerate}
\begin{table}[]
	\begin{tabular}{ccccc}
		Collision & Total M before & Total M after & Total Ke before & Total Ke after \\
		Col1      & 0.246          & 0.239         & 0.112           & 0.106          \\
		Col2      & -0.006         & -0.003        & 0.059           & 0.041          \\
		Col3      & 0.157          & 0.171         & 0.045           & 0.027          \\
		Col4      & 0.498          & -0.140        & 0.155           & 0.037          \\
		Col5      & 0.187          & 0.185         & 0.065           & 0.032
	\end{tabular}
\end{table}
\begin{table}[htpb]
	\centering
	\caption{Total Momentum and Kinetic Energy in all Five Collision Scenarios\label{col_calc}}
	\begin{tabular*}{\textwidth}{c@{\extracolsep{\fill}}ccccc}
		\toprule
		\textbf{\begin{tabular}[c]{@{}c@{}}Collision\\ Number\end{tabular}} & \textbf{\begin{tabular}[c]{@{}c@{}}Total $\bm{\vec{p}}$ Before\\ (kg $\bigcdot$ m/s)\end{tabular}} & \textbf{\begin{tabular}[c]{@{}c@{}}Total $\bm{\vec{p}}$ After\\(kg $\bigcdot$ m/s)\end{tabular}} & \textbf{\begin{tabular}[c]{@{}c@{}}Total $\bm{E_{k}}$\\ Before (J)\end{tabular}} & \textbf{\begin{tabular}[c]{@{}c@{}}Total $\bm{E_{k}}$\\ After (J)\end{tabular}} \\
		\midrule
		\textbf{Collision 1} & 0.246 & 0.239 & 0.112 & 0.106 \\\addlinespace[1.5mm]
		\textbf{Collision 2} & -0.006 & -0.003 & 0.059 & 0.041 \\\addlinespace[1.5mm]
		\textbf{Collision 3} & 0.157 & 0.171 & 0.045 & 0.027 \\\addlinespace[1.5mm]
		\textbf{Collision 4} & -0.157 & -0.140 & 0.155 & 0.037 \\\addlinespace[1.5mm]
		\textbf{Collision 5} & 0.187 & 0.185 & 0.065 & 0.032 \\ \bottomrule
	\end{tabular*}
\end{table}


\newpage
\section{Explosion Observations}
\textbf{The Two Explosion Scenarios:}
\vspace{-4mm}
\begin{enumerate}
	\itemsep-0.75em
	\item \textbf{Explosion 1:} Two carts of equal mass “explode” away from each other
	\item \textbf{Explosion 2:} Two carts of unequal mass “explode” away from each other
\end{enumerate}
% Please add the following required packages to your document preamble:
% \usepackage{booktabs}
\begin{table}[]
	\begin{tabular}{@{}ccccc@{}}
		Explosion & Total M before & Total M after & Total Ke before & Total Ke after \\
		Expo1     & 0.000          & 0.444         & 0.000           & 0.183          \\
		Expo2     & 0.000          & 0.524         & 0.000           & 0.189
	\end{tabular}
\end{table}
% Please add the following required packages to your document preamble:
% \usepackage{booktabs}
% \usepackage[table,xcdraw]{xcolor}
% If you use beamer only pass "xcolor=table" option, i.e. \documentclass[xcolor=table]{beamer}
\begin{table}[]
	\begin{tabular}{@{}ccccccccccc@{}}
		\toprule
		Explosion & Mass 1 & Mass 2 & Initial 1 & Initial 2 & Final 1 & Final 2 & Initial M 1 & Initial M 2 & Final M 1 & \cellcolor[HTML]{FFFFFF}Final M 2 \\ \midrule
		Expo1     & 0.270  & 0.270  & 0.000     & 0.000     & 0.796   & 0.848   & 0.000       & 0.000       & 0.215     & 0.229                             \\
		Expo2     & 0.270  & 0.532  & 0.000     & 0.000     & 0.945   & 0.506   & 0.000       & 0.000       & 0.255     & 0.269                             \\ \bottomrule
	\end{tabular}
\end{table}
\vspace{-6mm}
\section{Sample Calculations}
\vspace{-4mm}
\begin{minipage}[t]{0.3\textwidth}
	\underline{Collision 1 \bm{$\vec{p}_{i}$} Calculation:}
	\vspace{-8mm}
	\[
		\begin{aligned}
			\vec{p}_{i} & = m\vec{v}_{i}                           \\[-8pt]
			            & = 0.270(0.911)                           \\[-8pt]
			\vec{p}_{i} & = 0.246~\text{kg $\bigcdot$ m/s [right]}
		\end{aligned}
	\]
\end{minipage}%
\hspace{0.5cm}
\begin{minipage}[t]{0.3\textwidth}
	\underline{Collision 2 \bm{$\Sigma \vec{p}_{i}$} Calculation:}
	\vspace{-8mm}
	\[
		\begin{aligned}
			\Sigma\vec{p}_{i} & = \vec{p}_{i_{1}} + \vec{p}_{i_{2}}     \\[-8pt]
			                  & = m(v_{i_{1}} + v_{i_{2}})              \\[-8pt]
			                  & = 0.270(0.455 - 0.476)                  \\[-8pt]
			\Sigma\vec{p}_{i} & = 0.006~\text{kg $\bigcdot$ m/s [left]}
		\end{aligned}
	\]
\end{minipage}%
\hspace{0.5cm}
\begin{minipage}[t]{0.3\textwidth}
	\underline{Collision 2 \bm{$\Sigma E_{k_{i}}$} Calculation:}
	\vspace{-8mm}
	\[
		\begin{aligned}
			\Sigma E_{k_{i}} & = E_{k_{i_{1}}} + E_{k_{i_{2}}}            \\[-8pt]
			                 & = \sfrac{1}{2}m(v_{i_{1}}^2 + v_{i_{2}}^2) \\[-8pt]
			                 & = \sfrac{1}{2}0.270(0.455^2 + (-0.476)^2)  \\[-8pt]
			\Sigma E_{k_{i}} & = 0.059~\text{J}
		\end{aligned}
	\]
\end{minipage}%
\vspace{4mm}
\begin{minipage}[t]{0.3\textwidth}
	\underline{Explosion 2 \bm{$\vec{p}_{f}$} Calculation:}
	\vspace{-8mm}
	\[
		\begin{aligned}
			\vec{p}_{f} & = m\vec{v}_{f}                          \\[-8pt]
			            & = 0.532(-0.506)                         \\[-8pt]
			\vec{p}_{f} & = 0.269~\text{kg $\bigcdot$ m/s [left]}
		\end{aligned}
	\]
\end{minipage}%
\hspace{0.5cm}
\begin{minipage}[t]{0.3\textwidth}
	\underline{Explosion 1 \bm{$\Sigma \vec{p}_{f}$} Calculation:}
	\vspace{-8mm}
	\[
		\begin{aligned}
			\Sigma\vec{p}_{f} & = \vec{p}_{f_{1}} + \vec{p}_{f_{2}}     \\[-8pt]
			                  & = m(v_{f_{1}} + v_{f_{2}})              \\[-8pt]
			                  & = 0.270(0.796 - 0.848)                  \\[-8pt]
			\Sigma\vec{p}_{f} & = 0.014~\text{kg $\bigcdot$ m/s [left]}
		\end{aligned}
	\]
\end{minipage}%
\hspace{0.75cm}
\begin{minipage}[t]{0.3\textwidth}
	\underline{Explosion 1 \bm{$\Sigma E_{k_{f}}$} Calculation:}
	\vspace{-8mm}
	\[
		\begin{aligned}
			\Sigma E_{k_{f}} & = E_{k_{f_{1}}} + E_{k_{f_{2}}}            \\[-8pt]
			                 & = \sfrac{1}{2}m(v_{f_{1}}^2 + v_{f_{2}}^2) \\[-8pt]
			                 & = \sfrac{1}{2}0.270(0.796^2 + (-0.848)^2)  \\[-8pt]
			\Sigma E_{k_{f}} & = 0.183~\text{J}
		\end{aligned}
	\]
\end{minipage}%

\section{Collision Analysis}
\begin{enumerate}
	\item \textbf{Explain if your results show momentum being conserved after each of the 5 collisions.}\\
	      Looking at Table 3, it is evident that momentum is not perfectly conserved as the total momentum before is not equal to the total momentum after the collision.
	      However, taking a closer look at the table values, it reveals that while momentum is not perfectly conserved, it is objectively close to being conserved.
	      Additionally, the conservation of momentum applies to a closed system, which was not the case of the experiment as there were external interactions with the system.
	      The main reasons for why momentum is not conserved is due to friction, and external forces acting on the carts.
	      The equation for momentum ($\vec{p} = m\vec{v}$) shows that momentum is dependant on velocity, therefore changes in velocity directly cause a change in momentum.
	      As a result, the decrease in velocity due to friction causes a lower momentum, which is why momentum is not conserved.
	      The other factor is external forces acting on the carts which creates acceleration and a non zero net force.
	      When a cart is pushed, it is difficult to push in such a way that there is no acceleration.
	      Due to the acceleration, the velocity of the carts are altered again, where in this case their values increase the momentum, since the velocity increased.
	      The combination of the two factors cause momentum to not be conserved, however the data collected, still reveals that momentum is mostly conserved due to the low friction of the track and cart wheels.
	\item \textbf{Clasify each of the 5 collisions as elastic or inelastic.}\\
	      \underline{Collision 1:} The collision is elastic as both momentum and kinetic energy were mostly conserved.\\
	      \underline{Collision 2:} The collision is elastic as both momentum and kinetic energy were mostly conserved.\\
	      \underline{Collision 3:} The collision is inelastic because the kinetic energy was not conserved, and is on the spectrum between elastic and inelastic collision. (Special case)\\
	      \underline{Collision 4:} The collision is inelastic because the kinetic energy was not conserved, and is on the spectrum between elastic and inelastic collision. (Special case) \\
	      \underline{Collision 5:} The collision is perfectly inelastic as only momentum was conserved and the carts stuck together after collision, sharing the same final velocity.
	\item \textbf{What general trend, patterns or special cases do you see in your data regarding momentum or kinetic energy.}\\
	      A general trend for momentum was that when the blue cart was initially at rest, the red cart came to a standstill ($\vec{v}_{f} = 0$), which also means that the red cart transferred most of its momentum to the blue cart.
	      By theory this trend makes sense, because the blue cart starts at rest ($\vec{p}_{i} = 0$), which means for momentum to be conserved, the red cart must transfer all its momentum.
	      The reason for any momentum losses/gains is explained in question 1.
	      An observed trend for kinetic energy is that special cases of the collision being on the spectrum between elastic and inelastic only ocurred when using carts of different masses.
	      In both of those collisions (Collision 3 \& 4), the kinetic energy went down, which signifies that by the addition of mass to the second cart, magnetic damping could be evident.
	      The effect of the damping is that the kinetic energy is converted into heat energy during the collision through electric eddy currents.
	      The reason this effect is less prevalent in the first two collisions is because both the carts had the same mass, energy was transferred more ``cleanly".
	      Whereas with the blue cart having a greater mass (more inertia), the magnets get pushed closer, which makes it easier for the kinetic energy loss due to heat.
\end{enumerate}

\section{Explosion Analysis}
\begin{enumerate}
	\item \textbf{Describe what you notice about the total momentum of the system before and after the explosion.}\\
	      Since both carts started at rest their initial momentums were 0.
	      After the explosion, in both scenarios the final momentum was the same, but momentum was not conserved as explained previously.
	\item \textbf{Describe how the carts final velocities are different depending ont their masses. How would this relate to predicting the final velocities for a cannon ball and cannon, after the cannon is fired.}\\
	      The carts final velocities are dependant on their masses such that if an objects mass increases, its velocity decreases.
	      In the case of the second explosion scenario, since the blue cart was twice the mass of the red cart, its final velocity was around half of the red carts final velocity.
	      This proportional relationship could be applied to the cannon and cannon ball where finding the factor by which the cannon is heavier than the cannon ball.
	      By knowing the mass factor, cannon would have a final velocity of the cannon ball divided by the mass factor, or vice versa, where the cannon ball would have a final velocity of the cannon times the mass factor.
\end{enumerate}
\end{document}
