\documentclass[12pt,letterpaper]{article}
\newcommand\tab[1][1cm]{\hspace*{#1}}
\usepackage[utf8]{inputenc}
\usepackage[letterpaper]{geometry}
\usepackage{amsmath}
\usepackage{gensymb}
\usepackage{amssymb}
\usepackage{xcolor}
\usepackage{booktabs}
\usepackage{rotating}
\usepackage{fancyhdr}
\usepackage{hyperref}
\usepackage{setspace}
\usepackage{mathtools}
\usepackage{float}
\usepackage{caption}
\usepackage{bm}
\usepackage{enumitem}
\usepackage{pgfplots}
\usepackage{pgfplotstable}
\pgfplotsset{compat=newest}

\tolerance=1
\emergencystretch=\maxdimen
\hyphenpenalty=10000
\hbadness=10000

\captionsetup[table]{skip=0pt,singlelinecheck=off}
\doublespacing 

\geometry{top=0.6in, bottom=0.75in, left=1.0in, right=1.0in}
\renewcommand{\headrulewidth}{0pt} 
\renewcommand{\footrulewidth}{0pt} 
\setlength\headsep{0.333in}


\title{\textbf{Spring Force Activity}}
\author{\textbf{Siddharth Nema} \\Partner: Andy Zhang \\SPH4U0 | Ms.\hspace{-1mm} Ally}

\begin{document}
\maketitle
\newpage
\section{Observations}
\begin{table}[htpb]
	\centering
	\caption{Force and Displacement Data for Two Springs\label{table1}}

	\begin{tabular*}{0.9\textwidth}{l@{\extracolsep{\fill}}ccc}
		\toprule
		\textbf{Mass (kg)} &
		\textbf{Force (N)} &
		\textbf{\begin{tabular}[c]{@{}c@{}}Spring A\\ Displacement (cm)\end{tabular}} &
		\textbf{\begin{tabular}[c]{@{}c@{}}Spring B\\ Displacement (cm)\end{tabular}} \\ \midrule
		0.20 & 1.96 & 2.20 & 0.270 \\
		0.40 & 3.92 & 10.3 & 0.450 \\
		0.60 & 5.67 & 19.1 & 0.920 \\
		0.80 & 7.85 & 28.1 & 1.43  \\
		1.0  & 9.81 & 37.6 & 1.91  \\
		1.2  & 11.8 & 45.1 & 2.40  \\ \bottomrule
	\end{tabular*}
\end{table}

\vspace{-8mm}
\section{Analysis}
\vspace{-4mm}
\begin{figure}[H]
	\caption{Force Vs. Displacement for Spring One\label{figure1}}
	\vspace{-4mm}
	\begin{tikzpicture}
		\begin{axis}[
				xmin=0, xmax=40,
				ymin=0, ymax=12,
				xtick distance = 5,
				ytick distance = 2,
				xticklabel style={
						/pgf/number format/precision=3,
						/pgf/number format/fixed},
				grid = both,
				minor tick num = 1,
				major grid style = {lightgray},
				minor grid style = {lightgray!25},
				width = \textwidth,
				height = 0.5\textwidth,
				xlabel = {Displacement (cm)},
				ylabel = {Force (N)},
			]

			\addplot[red, only marks] file[skip first] {./Data/spring1.dat};
			\addplot [color=blue, domain=0:40, mark=none smooth, thick] {0.235*x+1.4};

		\end{axis}
	\end{tikzpicture}
\end{figure}
\begin{figure}[H]
	\caption{Force Vs. Displacement for Spring Two\label{figure2}}
	\vspace{-4mm}
	\begin{tikzpicture}
		\begin{axis}[
				xmin=0, xmax=5,
				ymin=0, ymax=25,
				xtick distance = 1,
				ytick distance = 5,
				xticklabel style={
						/pgf/number format/precision=3,
						/pgf/number format/fixed},
				grid = both,
				minor tick num = 1,
				major grid style = {lightgray},
				minor grid style = {lightgray!25},
				width = \textwidth,
				height = 0.5\textwidth,
				xlabel = {Displacement (cm)},
				ylabel = {Force (N)},
			]

			\addplot[red, only marks] file[skip first] {./Data/spring2.dat};
			\addplot [color=blue, domain=0:5, mark=none smooth, thick] {3.86*x+2.21};

		\end{axis}
	\end{tikzpicture}
\end{figure}

\newpage

\begin{enumerate}[font=\bfseries]
	\item \textbf{Calculating Slope of Each Graph:}
	      \vspace{-4mm}
	      \begin{enumerate}
		      \item \underline{Spring A}\\
		            The trendline's equation for Spring A is:~\bm{$y = 0.225x + 1.48$}\\
		            $\therefore$ The slope of trendline is 0.225.
		      \item \underline{Spring B}\\
		            The trendline's equation for Spring B is:~\bm{$y = 4.38x + 1.44$}\\
		            $\therefore$ The slope of trendline is 4.38.
	      \end{enumerate}
	\item \textbf{Relationship Between Force and Displacement:}\\
	      There is a linear relationship between force and displacement, where force is directly proportional to displacement ($\vec{F} \propto \vec{\Delta d}$).
	\item \textbf{Physical Quantity the Slope Represents:}\\
	      The physical quantity the slope represents is the spring constant, which determines the stiffness of the spring. The two springs have different values because each spring has a different spring constant, where a higher slope value represents a stiffer spring. This holds true because Spring B has a larger slope than Spring A, meaning it is more stiff, evident through Spring B's smaller displacement values.
	\item \textbf{Equation of Line from Proportionality Statement:}\\
	      Let $k$ represent the slope of the line.\\
	      $\vec{F} = k\vec{\Delta d}$
	\item \textbf{Properties of an "Ideal Spring":}
\end{enumerate}

\end{document}