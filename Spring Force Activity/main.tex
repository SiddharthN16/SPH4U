\documentclass[12pt,letterpaper]{article}

\newcommand\tab[1][1cm]{\hspace*{#1}}
\usepackage[utf8]{inputenc}
\usepackage[letterpaper]{geometry}
\usepackage{amsmath}
\usepackage{gensymb}
\usepackage{amssymb}
\usepackage{xcolor}
\usepackage{booktabs}
\usepackage{rotating}
\usepackage{fancyhdr}
\usepackage{hyperref}
\usepackage{setspace}
\usepackage{mathtools}
\usepackage{float}
\usepackage{caption}
\usepackage{bm}
\usepackage{enumitem}
\usepackage{pgfplots}
\usepackage{pgfplotstable}

\usepackage{physics}
\usepackage{tikz}
\usepackage{mathdots}
\usepackage{yhmath}
\usepackage{cancel}
\usepackage{color}
\usepackage{siunitx}
\usepackage{array}
\usepackage{multirow}
\usepackage{tabularx}
\usepackage{extarrows}
\usetikzlibrary{fadings}
\usetikzlibrary{patterns}
\usetikzlibrary{shadows.blur}
\usetikzlibrary{shapes}
\pgfplotsset{compat=newest}

\tolerance=1
\emergencystretch=\maxdimen
\hyphenpenalty=10000
\hbadness=10000

\captionsetup[table]{skip=0pt,singlelinecheck=off}
\doublespacing 

\geometry{top=0.6in, bottom=0.75in, left=1.0in, right=1.0in}
\renewcommand{\headrulewidth}{0pt} 
\renewcommand{\footrulewidth}{0pt} 
\setlength\headsep{0.333in}


\title{\textbf{Spring Force Activity}}
\author{\textbf{Siddharth Nema} \\Partner: Andy Zhang \\SPH4U0 | Ms.\hspace{-1mm} Ally}

\begin{document}
\maketitle
\newpage
\section{Setup}
\centering

% Gradient Info

\tikzset {_hgvizh62w/.code = {\pgfsetadditionalshadetransform{ \pgftransformshift{\pgfpoint{0 bp } { 45 bp }  }  \pgftransformscale{3 }  }}}
\pgfdeclareradialshading{_2r7j8qrt6}{\pgfpoint{0bp}{-16bp}}{rgb(0bp)=(0.91,0.88,0.88);
	rgb(0bp)=(0.91,0.88,0.88);
	rgb(12.678571428571427bp)=(0.74,0.73,0.73);
	rgb(25bp)=(0.74,0.73,0.73);
	rgb(400bp)=(0.74,0.73,0.73)}

% Gradient Info

\tikzset {_cbuxahtvq/.code = {\pgfsetadditionalshadetransform{ \pgftransformshift{\pgfpoint{0 bp } { 0 bp }  }  \pgftransformscale{3 }  }}}
\pgfdeclareradialshading{_dspv5dbg6}{\pgfpoint{0bp}{0bp}}{rgb(0bp)=(0.9,0.89,0.89);
	rgb(0bp)=(0.9,0.89,0.89);
	rgb(14.196428571428571bp)=(0.8,0.77,0.77);
	rgb(25bp)=(0.74,0.72,0.72);
	rgb(400bp)=(0.74,0.72,0.72)}

% Gradient Info

\tikzset {_9ugufig8u/.code = {\pgfsetadditionalshadetransform{ \pgftransformshift{\pgfpoint{116.1 bp } { -141.9 bp }  }  \pgftransformscale{1.72 }  }}}
\pgfdeclareradialshading{_7m8avh0cx}{\pgfpoint{-72bp}{88bp}}{rgb(0bp)=(0.9,0.89,0.89);
	rgb(0bp)=(0.9,0.89,0.89);
	rgb(4.196428571428571bp)=(0.9,0.89,0.89);
	rgb(8.303571428571429bp)=(0.9,0.89,0.89);
	rgb(12.857142857142856bp)=(0.93,0.89,0.89);
	rgb(25bp)=(0.67,0.66,0.66);
	rgb(400bp)=(0.67,0.66,0.66)}

% Gradient Info

\tikzset {_g11d4vbjs/.code = {\pgfsetadditionalshadetransform{ \pgftransformshift{\pgfpoint{0 bp } { 0 bp }  }  \pgftransformscale{1.38 }  }}}
\pgfdeclareradialshading{_usxej9iod}{\pgfpoint{0bp}{0bp}}{rgb(0bp)=(0.85,0.85,0.85);
	rgb(0bp)=(0.85,0.85,0.85);
	rgb(11.875bp)=(0.83,0.82,0.82);
	rgb(25bp)=(0.69,0.68,0.68);
	rgb(400bp)=(0.69,0.68,0.68)}

% Gradient Info

\tikzset {_u6002ew9r/.code = {\pgfsetadditionalshadetransform{ \pgftransformshift{\pgfpoint{0 bp } { 0 bp }  }  \pgftransformscale{3 }  }}}
\pgfdeclareradialshading{_l58m1p03t}{\pgfpoint{0bp}{0bp}}{rgb(0bp)=(0.9,0.89,0.89);
	rgb(0bp)=(0.9,0.89,0.89);
	rgb(14.196428571428571bp)=(0.8,0.77,0.77);
	rgb(25bp)=(0.74,0.72,0.72);
	rgb(400bp)=(0.74,0.72,0.72)}

% Gradient Info

\tikzset {_778d7p0wp/.code = {\pgfsetadditionalshadetransform{ \pgftransformshift{\pgfpoint{0 bp } { 0 bp }  }  \pgftransformscale{1.38 }  }}}
\pgfdeclareradialshading{_yhcsr4afw}{\pgfpoint{0bp}{0bp}}{rgb(0bp)=(0.85,0.85,0.85);
	rgb(0bp)=(0.85,0.85,0.85);
	rgb(11.875bp)=(0.83,0.82,0.82);
	rgb(25bp)=(0.69,0.68,0.68);
	rgb(400bp)=(0.69,0.68,0.68)}
\tikzset{every picture/.style={line width=0.75pt}} %set default line width to 0.75pt        

\begin{tikzpicture}[x=0.75pt,y=0.75pt,yscale=-1,xscale=1]
	%uncomment if require: \path (0,480); %set diagram left start at 0, and has height of 480

	%Shape: Rectangle [id:dp661407400810794] 
	\draw  [fill={rgb, 255:red, 146; green, 108; blue, 75 }  ,fill opacity=1 ] (137.34,380.7) -- (369.11,380.7) -- (369.11,419.99) -- (137.34,419.99) -- cycle ;
	%Rounded Rect [id:dp14042717508972213] 
	\path  [shading=_2r7j8qrt6,_hgvizh62w] (356.84,65.35) .. controls (356.84,64.05) and (357.89,63) .. (359.19,63) -- (361.15,63) .. controls (362.45,63) and (363.5,64.05) .. (363.5,65.35) -- (363.5,369.65) .. controls (363.5,370.95) and (362.45,372) .. (361.15,372) -- (359.19,372) .. controls (357.89,372) and (356.84,370.95) .. (356.84,369.65) -- cycle ; % for fading 
	\draw   (356.84,65.35) .. controls (356.84,64.05) and (357.89,63) .. (359.19,63) -- (361.15,63) .. controls (362.45,63) and (363.5,64.05) .. (363.5,65.35) -- (363.5,369.65) .. controls (363.5,370.95) and (362.45,372) .. (361.15,372) -- (359.19,372) .. controls (357.89,372) and (356.84,370.95) .. (356.84,369.65) -- cycle ; % for border 

	%Shape: Spring [id:dp16658294194788303] 
	\draw  [color={rgb, 255:red, 74; green, 74; blue, 74 }  ,draw opacity=1 ][line width=0.75]  (438.84,87.06) .. controls (442.52,87.55) and (446.21,89.52) .. (446.21,93.44) .. controls (446.21,101.3) and (431.47,101.3) .. (431.47,97.37) .. controls (431.47,93.44) and (446.21,93.44) .. (446.21,101.3) .. controls (446.21,109.16) and (431.47,109.16) .. (431.47,105.23) .. controls (431.47,101.3) and (446.21,101.3) .. (446.21,109.16) .. controls (446.21,117.01) and (431.47,117.01) .. (431.47,113.09) .. controls (431.47,109.16) and (446.21,109.16) .. (446.21,117.01) .. controls (446.21,124.87) and (431.47,124.87) .. (431.47,120.94) .. controls (431.47,117.01) and (446.21,117.01) .. (446.21,124.87) .. controls (446.21,132.73) and (431.47,132.73) .. (431.47,128.8) .. controls (431.47,124.87) and (446.21,124.87) .. (446.21,132.73) .. controls (446.21,140.58) and (431.47,140.58) .. (431.47,136.66) .. controls (431.47,132.73) and (446.21,132.73) .. (446.21,140.58) .. controls (446.21,148.44) and (431.47,148.44) .. (431.47,144.51) .. controls (431.47,140.58) and (446.21,140.58) .. (446.21,148.44) .. controls (446.21,156.3) and (431.47,156.3) .. (431.47,152.37) .. controls (431.47,148.44) and (446.21,148.44) .. (446.21,156.3) .. controls (446.21,164.15) and (431.47,164.15) .. (431.47,160.23) .. controls (431.47,156.3) and (446.21,156.3) .. (446.21,164.15) .. controls (446.21,172.01) and (431.47,172.01) .. (431.47,168.08) .. controls (431.47,164.15) and (446.21,164.15) .. (446.21,172.01) .. controls (446.21,179.87) and (431.47,179.87) .. (431.47,175.94) .. controls (431.47,172.01) and (446.21,172.01) .. (446.21,179.87) .. controls (446.21,187.72) and (431.47,187.72) .. (431.47,183.8) .. controls (431.47,179.87) and (446.21,179.87) .. (446.21,187.72) .. controls (446.21,195.58) and (431.47,195.58) .. (431.47,191.65) .. controls (431.47,187.72) and (446.21,187.72) .. (446.21,195.58) .. controls (446.21,199.51) and (442.52,201.47) .. (438.84,201.96) ;
	%Shape: Rectangle [id:dp22739131790719957] 
	\path  [shading=_dspv5dbg6,_cbuxahtvq] (340.72,84.94) -- (346.77,84.94) -- (346.77,86.73) -- (340.72,86.73) -- cycle ; % for fading 
	\draw  [color={rgb, 255:red, 0; green, 0; blue, 0 }  ,draw opacity=1 ] (340.72,84.94) -- (346.77,84.94) -- (346.77,86.73) -- (340.72,86.73) -- cycle ; % for border 

	%Rounded Rect [id:dp8164684948783012] 
	\path  [shading=_7m8avh0cx,_9ugufig8u] (365.18,84.51) .. controls (365.18,84.02) and (365.58,83.62) .. (366.07,83.62) -- (456.62,83.62) .. controls (457.1,83.62) and (457.5,84.02) .. (457.5,84.51) -- (457.5,87.16) .. controls (457.5,87.65) and (457.1,88.04) .. (456.62,88.04) -- (366.07,88.04) .. controls (365.58,88.04) and (365.18,87.65) .. (365.18,87.16) -- cycle ; % for fading 
	\draw   (365.18,84.51) .. controls (365.18,84.02) and (365.58,83.62) .. (366.07,83.62) -- (456.62,83.62) .. controls (457.1,83.62) and (457.5,84.02) .. (457.5,84.51) -- (457.5,87.16) .. controls (457.5,87.65) and (457.1,88.04) .. (456.62,88.04) -- (366.07,88.04) .. controls (365.58,88.04) and (365.18,87.65) .. (365.18,87.16) -- cycle ; % for border 

	%Rounded Same Side Corner Rect [id:dp3337767177690609] 
	\draw  [fill={rgb, 255:red, 236; green, 148; blue, 2 }  ,fill opacity=1 ] (425.36,225.8) .. controls (425.36,220.23) and (429.87,215.71) .. (435.44,215.71) -- (444.24,215.71) .. controls (449.81,215.71) and (454.33,220.23) .. (454.33,225.8) -- (454.33,246.09) .. controls (454.33,246.09) and (454.33,246.09) .. (454.33,246.09) -- (425.36,246.09) .. controls (425.36,246.09) and (425.36,246.09) .. (425.36,246.09) -- cycle ;
	%Straight Lines [id:da9036000547278182] 
	\draw [color={rgb, 255:red, 0; green, 0; blue, 0 }  ,draw opacity=1 ]   (439.82,215.22) -- (439.82,200.98) ;
	%Rounded Rect [id:dp8364710468830279] 
	\path  [shading=_usxej9iod,_g11d4vbjs] (347.44,81.14) .. controls (347.44,79.59) and (348.69,78.34) .. (350.24,78.34) -- (372.21,78.34) .. controls (373.75,78.34) and (375.01,79.59) .. (375.01,81.14) -- (375.01,89.53) .. controls (375.01,91.07) and (373.75,92.32) .. (372.21,92.32) -- (350.24,92.32) .. controls (348.69,92.32) and (347.44,91.07) .. (347.44,89.53) -- cycle ; % for fading 
	\draw  [color={rgb, 255:red, 74; green, 74; blue, 74 }  ,draw opacity=1 ] (347.44,81.14) .. controls (347.44,79.59) and (348.69,78.34) .. (350.24,78.34) -- (372.21,78.34) .. controls (373.75,78.34) and (375.01,79.59) .. (375.01,81.14) -- (375.01,89.53) .. controls (375.01,91.07) and (373.75,92.32) .. (372.21,92.32) -- (350.24,92.32) .. controls (348.69,92.32) and (347.44,91.07) .. (347.44,89.53) -- cycle ; % for border 

	%Shape: Ellipse [id:dp5179670735428143] 
	\draw  [fill={rgb, 255:red, 0; green, 0; blue, 0 }  ,fill opacity=1 ] (331.3,84.94) .. controls (331.3,80.52) and (333.41,76.93) .. (336.01,76.93) .. controls (338.61,76.93) and (340.72,80.52) .. (340.72,84.94) .. controls (340.72,89.37) and (338.61,92.95) .. (336.01,92.95) .. controls (333.41,92.95) and (331.3,89.37) .. (331.3,84.94) -- cycle ;
	%Shape: Rectangle [id:dp7277974618781018] 
	\path  [shading=_l58m1p03t,_u6002ew9r] (339.24,125.7) -- (345.29,125.7) -- (345.29,127.49) -- (339.24,127.49) -- cycle ; % for fading 
	\draw  [color={rgb, 255:red, 0; green, 0; blue, 0 }  ,draw opacity=1 ] (339.24,125.7) -- (345.29,125.7) -- (345.29,127.49) -- (339.24,127.49) -- cycle ; % for border 

	%Rounded Rect [id:dp9330549543329489] 
	\path  [shading=_yhcsr4afw,_778d7p0wp] (345.97,121.89) .. controls (345.97,120.35) and (347.22,119.1) .. (348.76,119.1) -- (370.74,119.1) .. controls (372.28,119.1) and (373.53,120.35) .. (373.53,121.89) -- (373.53,130.28) .. controls (373.53,131.83) and (372.28,133.08) .. (370.74,133.08) -- (348.76,133.08) .. controls (347.22,133.08) and (345.97,131.83) .. (345.97,130.28) -- cycle ; % for fading 
	\draw  [color={rgb, 255:red, 74; green, 74; blue, 74 }  ,draw opacity=1 ] (345.97,121.89) .. controls (345.97,120.35) and (347.22,119.1) .. (348.76,119.1) -- (370.74,119.1) .. controls (372.28,119.1) and (373.53,120.35) .. (373.53,121.89) -- (373.53,130.28) .. controls (373.53,131.83) and (372.28,133.08) .. (370.74,133.08) -- (348.76,133.08) .. controls (347.22,133.08) and (345.97,131.83) .. (345.97,130.28) -- cycle ; % for border 

	%Shape: Ellipse [id:dp27429498112957074] 
	\draw  [fill={rgb, 255:red, 0; green, 0; blue, 0 }  ,fill opacity=1 ] (329.83,125.7) .. controls (329.83,121.28) and (331.94,117.69) .. (334.54,117.69) .. controls (337.14,117.69) and (339.24,121.28) .. (339.24,125.7) .. controls (339.24,130.12) and (337.14,133.71) .. (334.54,133.71) .. controls (331.94,133.71) and (329.83,130.12) .. (329.83,125.7) -- cycle ;
	%Straight Lines [id:da6364752459654412] 
	\draw [color={rgb, 255:red, 110; green, 61; blue, 18 }  ,draw opacity=1 ][line width=0.75]    (373.53,125.85) -- (380.9,125.85) ;
	%Shape: Rectangle [id:dp879583802615739] 
	\draw  [color={rgb, 255:red, 121; green, 58; blue, 3 }  ,draw opacity=1 ][fill={rgb, 255:red, 97; green, 47; blue, 4 }  ,fill opacity=1 ] (380.41,122.91) -- (380.9,122.91) -- (380.9,127.82) -- (380.41,127.82) -- cycle ;
	%Shape: Rectangle [id:dp2590872105368831] 
	\draw  [color={rgb, 255:red, 0; green, 0; blue, 0 }  ,draw opacity=1 ][fill={rgb, 255:red, 238; green, 188; blue, 105 }  ,fill opacity=1 ][line width=0.75]  (403.49,420.97) -- (382.86,420.97) -- (382.86,88.53) -- (403.49,88.53) -- cycle ;
	%Straight Lines [id:da4342154910571179] 
	\draw    (403.49,408.51) -- (390.72,408.51) ;
	%Straight Lines [id:da8960085846992045] 
	\draw    (403.49,398.04) -- (390.72,398.04) ;
	%Straight Lines [id:da29417082715176046] 
	\draw    (403.49,388.57) -- (390.72,388.57) ;
	%Straight Lines [id:da7043382052247802] 
	\draw    (403.49,378.6) -- (390.72,378.6) ;
	%Straight Lines [id:da14461348672257412] 
	\draw    (403.49,368.64) -- (390.72,368.64) ;
	%Straight Lines [id:da8812764216308848] 
	\draw    (403.49,358.67) -- (390.72,358.67) ;
	%Straight Lines [id:da46113320559838833] 
	\draw    (403.49,348.7) -- (390.72,348.7) ;
	%Straight Lines [id:da06870609801527894] 
	\draw    (403.49,338.73) -- (390.72,338.73) ;
	%Straight Lines [id:da6491635747766802] 
	\draw    (403.49,328.76) -- (390.72,328.76) ;
	%Straight Lines [id:da6541158039132522] 
	\draw    (403.49,318.8) -- (390.72,318.8) ;
	%Straight Lines [id:da8229925427369018] 
	\draw    (403.49,308.83) -- (390.72,308.83) ;
	%Straight Lines [id:da9289832823167388] 
	\draw    (403.49,298.86) -- (390.72,298.86) ;
	%Straight Lines [id:da1480662946592668] 
	\draw    (403.49,288.89) -- (390.72,288.89) ;
	%Straight Lines [id:da8076788219370534] 
	\draw    (403.49,280.92) -- (390.72,280.92) ;
	%Straight Lines [id:da7862953107859727] 
	\draw    (403.49,270.95) -- (390.72,270.95) ;
	%Straight Lines [id:da3509670935278919] 
	\draw    (403.49,260.98) -- (390.72,260.98) ;
	%Straight Lines [id:da9062043984578614] 
	\draw    (403.49,251.01) -- (390.72,251.01) ;
	%Straight Lines [id:da636865271801506] 
	\draw    (403.49,241.04) -- (390.72,241.04) ;
	%Straight Lines [id:da883281904699244] 
	\draw    (403.49,231.08) -- (390.72,231.08) ;
	%Straight Lines [id:da2697651405327248] 
	\draw    (403.49,221.11) -- (390.72,221.11) ;
	%Straight Lines [id:da01060565079167164] 
	\draw    (403.49,211.14) -- (390.72,211.14) ;
	%Straight Lines [id:da049150025088934735] 
	\draw    (403.49,201.17) -- (390.72,201.17) ;
	%Straight Lines [id:da7848421068801716] 
	\draw    (403.49,191.2) -- (390.72,191.2) ;
	%Straight Lines [id:da6766645206515829] 
	\draw    (403.49,181.24) -- (390.72,181.24) ;
	%Straight Lines [id:da3941559797200511] 
	\draw    (403.49,171.27) -- (390.72,171.27) ;
	%Straight Lines [id:da6345888877595267] 
	\draw    (403.49,161.3) -- (390.72,161.3) ;
	%Straight Lines [id:da5187847923306577] 
	\draw    (403.49,151.33) -- (390.72,151.33) ;
	%Straight Lines [id:da9569200064358407] 
	\draw    (403.49,141.36) -- (390.72,141.36) ;
	%Straight Lines [id:da6246783306251273] 
	\draw    (403.49,131.4) -- (390.72,131.4) ;
	%Straight Lines [id:da3271985919064788] 
	\draw    (403.49,121.43) -- (390.72,121.43) ;
	%Straight Lines [id:da0483997763552233] 
	\draw    (403.49,111.46) -- (390.72,111.46) ;
	%Straight Lines [id:da15577476855891215] 
	\draw    (403.49,101.49) -- (390.72,101.49) ;

	%Shape: Rectangle [id:dp17328219705210146] 
	\draw  [color={rgb, 255:red, 121; green, 58; blue, 3 }  ,draw opacity=1 ][fill={rgb, 255:red, 97; green, 47; blue, 4 }  ,fill opacity=1 ] (404.96,122.91) -- (405.45,122.91) -- (405.45,127.82) -- (404.96,127.82) -- cycle ;
	%Shape: Rectangle [id:dp9738368237348072] 
	\draw  [fill={rgb, 255:red, 74; green, 74; blue, 74 }  ,fill opacity=1 ] (258.63,367) -- (369.11,367) -- (369.11,380.7) -- (258.63,380.7) -- cycle ;

	% Text Node
	\draw (230.5,394.5) node [anchor=north west][inner sep=0.75pt]   [align=left] {{\fontfamily{pcr}\selectfont \textcolor[rgb]{1,0.92,0}{TABLE}}};
	% Text Node
	\draw (270,241) node [anchor=north west][inner sep=0.75pt]   [align=left] {{\fontfamily{pcr}\selectfont Ring Stand}};
	% Text Node
	\draw (234.5,116) node [anchor=north west][inner sep=0.75pt]   [align=left] {{\fontfamily{pcr}\selectfont Ruler Clamp}};
	% Text Node
	\draw (375.5,63.5) node [anchor=north west][inner sep=0.75pt]   [align=left] {{\fontfamily{pcr}\selectfont Spring Clamp}};
	% Text Node
	\draw (454.5,138.5) node [anchor=north west][inner sep=0.75pt]   [align=left] {{\fontfamily{pcr}\selectfont Spring}};
	% Text Node
	\draw (409.5,303) node [anchor=north west][inner sep=0.75pt]   [align=left] {{\fontfamily{pcr}\selectfont Ruler}};
	% Text Node
	\draw (459.5,224.5) node [anchor=north west][inner sep=0.75pt]   [align=left] {{\fontfamily{pcr}\selectfont Known Mass}};
	% Text Node
	\draw (427.5,227.5) node [anchor=north west][inner sep=0.75pt]   [align=left] {{\fontfamily{pcr}\selectfont {\scriptsize 500g}}};


\end{tikzpicture}

\section{Observations}
\begin{table}[htpb]
	\centering
	\caption{Force and Displacement Data for Two Springs\label{table1}}

	\begin{tabular*}{0.9\textwidth}{c@{\extracolsep{\fill}}ccc}
		\toprule
		\textbf{Mass (kg)} &
		\textbf{Force (N)} &
		\textbf{\begin{tabular}[c]{@{}c@{}}Spring A\\ Displacement (cm)\end{tabular}} &
		\textbf{\begin{tabular}[c]{@{}c@{}}Spring B\\ Displacement (cm)\end{tabular}} \\ \midrule
		0.20 & 1.96 & 2.20 & 0.270 \\
		0.40 & 3.92 & 10.3 & 0.450 \\
		0.60 & 5.67 & 19.1 & 0.920 \\
		0.80 & 7.85 & 28.1 & 1.43  \\
		1.0  & 9.81 & 37.6 & 1.91  \\
		1.2  & 11.8 & 45.1 & 2.40  \\ \bottomrule
	\end{tabular*}
\end{table}

\newpage

\section{Analysis}
\vspace{-4mm}
\begin{figure}[H]
	\caption{Spring 1: Force Vs. Displacement\label{figure1}}
	\vspace{-4mm}
	\begin{tikzpicture}
		\begin{axis}[
				xmin=0, xmax=40,
				ymin=0, ymax=12,
				xtick distance = 5,
				ytick distance = 2,
				xticklabel style={
						/pgf/number format/precision=3,
						/pgf/number format/fixed},
				grid = both,
				minor tick num = 1,
				major grid style = {lightgray},
				minor grid style = {lightgray!25},
				width = \textwidth,
				height = 0.5\textwidth,
				xlabel = {Displacement (m)},
				ylabel = {Force (N)},
			]

			\addplot[red, only marks] file[skip first] {./Data/spring1.dat};
			\addplot [color=blue, domain=0:0.3, mark=none smooth, thick] {7.5563*x-0.2134};

		\end{axis}
	\end{tikzpicture}
\end{figure}
\begin{figure}[H]
	\caption{Spring 2: Force Vs. Displacement\label{figure2}}
	\vspace{-4mm}
	\begin{tikzpicture}
		\begin{axis}[
				xmin=0, xmax=20,
				ymin=0, ymax=5,
				xtick distance = 2,
				ytick distance = 1,
				xticklabel style={
						/pgf/number format/precision=3,
						/pgf/number format/fixed},
				grid = both,
				minor tick num = 1,
				major grid style = {lightgray},
				minor grid style = {lightgray!25},
				width = \textwidth,
				height = 0.5\textwidth,
				xlabel = {Displacement (m)},
				ylabel = {Force(N)},
			]

			\addplot[red, only marks] file[skip first] {./data/spring2.dat};
			\addplot [color=blue, domain=0:0.3, mark=none smooth, thick] {7.5563*x-0.2134};

		\end{axis}
	\end{tikzpicture}
\end{figure}

\begin{enumerate}[font=\bfseries]
	\item \textbf{Calculating Slope of Each Graph:}
	      \vspace{-4mm}
	      \begin{enumerate}
		      \item \underline{Spring A}\\
		            The trendline's equation for Spring A is:~\bm{$y = 0.225x + 1.48$}\\
		            $\therefore$ The slope of the trendline is 0.225.
		      \item \underline{Spring B}\\
		            The trendline's equation for Spring B is:~\bm{$y = 4.38x + 1.44$}\\
		            $\therefore$ The slope of the trendline is 4.38.
	      \end{enumerate}
	      \newpage
	\item \textbf{Relationship Between Force and Displacement:}\\
	      There is a linear relationship between force and displacement, where force is directly proportional to displacement ($\vec{F} \propto \vec{\Delta d}$).
	\item \textbf{Physical Quantity the Slope Represents:}\\
	      The physical quantity the slope represents is the spring constant, which determines the stiffness of the spring. The two springs have different values because each spring has a different spring constant, where a higher slope value represents a stiffer spring. This holds true because Spring B has a larger slope than Spring A, meaning it is more stiff, evident through Spring B's smaller displacement values than Spring A.
	\item \textbf{Equation of Line from Proportionality Statement:}\\
	      Let $k$ represent the slope of the line.\\
	      $\vec{F} = k\vec{\Delta d}$
	\item \textbf{Properties of an "Ideal Spring":}
\end{enumerate}

\end{document}