\documentclass[12pt,letterpaper]{article}

\newcommand\tab[1][1cm]{\hspace*{#1}}
\usepackage[utf8]{inputenc}
\usepackage[letterpaper]{geometry}
\usepackage{amsmath}
\usepackage{gensymb}
\usepackage{amssymb}
\usepackage{xcolor}
\usepackage{booktabs}
\usepackage{rotating}
\usepackage{fancyhdr}
\usepackage{hyperref}
\usepackage{setspace}
\usepackage{mathtools}
\usepackage{float}
\usepackage{caption}
\usepackage{bm}
\usepackage{enumitem}
\usepackage{pgfplots}
\usepackage{pgfplotstable}

\usepackage{physics}
\usepackage{tikz}
\usepackage{mathdots}
\usepackage{yhmath}
\usepackage{cancel}
\usepackage{color}
\usepackage{siunitx}
\usepackage{array}
\usepackage{multirow}
\usepackage{tabularx}
\usepackage{extarrows}
\usetikzlibrary{fadings}
\usetikzlibrary{patterns}
\usetikzlibrary{shadows.blur}
\usetikzlibrary{shapes}
\pgfplotsset{compat=newest}

\tolerance=1
\emergencystretch=\maxdimen
\hyphenpenalty=10000
\hbadness=10000

\definecolor{dark-green}{HTML}{2C8243}

\captionsetup[table]{skip=0pt,singlelinecheck=off}
\doublespacing 

\geometry{top=0.6in, bottom=0.75in, left=1.0in, right=1.0in}
\renewcommand{\headrulewidth}{0pt} 
\renewcommand{\footrulewidth}{0pt} 
\setlength\headsep{0.333in}


\title{\textbf{Spring Force Activity}}
\author{\textbf{Siddharth Nema}\\SPH4U0 \\Ms.\hspace{-1mm} Ally}

\begin{document}
\maketitle
\newpage
\section{Experimental Setup}
% Gradient Info

\tikzset {_audied9e7/.code = {\pgfsetadditionalshadetransform{ \pgftransformshift{\pgfpoint{0 bp } { 45 bp }  }  \pgftransformscale{3 }  }}}
\pgfdeclareradialshading{_l1th8ezcg}{\pgfpoint{0bp}{-16bp}}{rgb(0bp)=(0.91,0.88,0.88);
	rgb(0bp)=(0.91,0.88,0.88);
	rgb(12.678571428571427bp)=(0.74,0.73,0.73);
	rgb(25bp)=(0.74,0.73,0.73);
	rgb(400bp)=(0.74,0.73,0.73)}

% Gradient Info

\tikzset {_zw799en1h/.code = {\pgfsetadditionalshadetransform{ \pgftransformshift{\pgfpoint{0 bp } { 0 bp }  }  \pgftransformscale{3 }  }}}
\pgfdeclareradialshading{_7fov45wgl}{\pgfpoint{0bp}{0bp}}{rgb(0bp)=(0.9,0.89,0.89);
	rgb(0bp)=(0.9,0.89,0.89);
	rgb(14.196428571428571bp)=(0.8,0.77,0.77);
	rgb(25bp)=(0.74,0.72,0.72);
	rgb(400bp)=(0.74,0.72,0.72)}

% Gradient Info

\tikzset {_ma7ryb15k/.code = {\pgfsetadditionalshadetransform{ \pgftransformshift{\pgfpoint{116.1 bp } { -141.9 bp }  }  \pgftransformscale{1.72 }  }}}
\pgfdeclareradialshading{_l971no8s8}{\pgfpoint{-72bp}{88bp}}{rgb(0bp)=(0.9,0.89,0.89);
	rgb(0bp)=(0.9,0.89,0.89);
	rgb(4.196428571428571bp)=(0.9,0.89,0.89);
	rgb(8.303571428571429bp)=(0.9,0.89,0.89);
	rgb(12.857142857142856bp)=(0.93,0.89,0.89);
	rgb(25bp)=(0.67,0.66,0.66);
	rgb(400bp)=(0.67,0.66,0.66)}

% Gradient Info

\tikzset {_f6tcg567o/.code = {\pgfsetadditionalshadetransform{ \pgftransformshift{\pgfpoint{0 bp } { 0 bp }  }  \pgftransformscale{1.38 }  }}}
\pgfdeclareradialshading{_u0fs4tjoa}{\pgfpoint{0bp}{0bp}}{rgb(0bp)=(0.85,0.85,0.85);
	rgb(0bp)=(0.85,0.85,0.85);
	rgb(11.875bp)=(0.83,0.82,0.82);
	rgb(25bp)=(0.69,0.68,0.68);
	rgb(400bp)=(0.69,0.68,0.68)}

% Gradient Info

\tikzset {_kvu3t6wn8/.code = {\pgfsetadditionalshadetransform{ \pgftransformshift{\pgfpoint{0 bp } { 0 bp }  }  \pgftransformscale{3 }  }}}
\pgfdeclareradialshading{_ifkxbtcrd}{\pgfpoint{0bp}{0bp}}{rgb(0bp)=(0.9,0.89,0.89);
	rgb(0bp)=(0.9,0.89,0.89);
	rgb(14.196428571428571bp)=(0.8,0.77,0.77);
	rgb(25bp)=(0.74,0.72,0.72);
	rgb(400bp)=(0.74,0.72,0.72)}

% Gradient Info

\tikzset {_relekxdrh/.code = {\pgfsetadditionalshadetransform{ \pgftransformshift{\pgfpoint{0 bp } { 0 bp }  }  \pgftransformscale{1.38 }  }}}
\pgfdeclareradialshading{_i9ec3u4oc}{\pgfpoint{0bp}{0bp}}{rgb(0bp)=(0.85,0.85,0.85);
	rgb(0bp)=(0.85,0.85,0.85);
	rgb(11.875bp)=(0.83,0.82,0.82);
	rgb(25bp)=(0.69,0.68,0.68);
	rgb(400bp)=(0.69,0.68,0.68)}
\tikzset{every picture/.style={line width=0.75pt}} %set default line width to 0.75pt        

\begin{figure}[H]
	\centering
	\begin{tikzpicture}[x=0.75pt,y=0.75pt,yscale=-1,xscale=1]

		%uncomment if require: \path (0,480); %set diagram left start at 0, and has height of 480

		%Shape: Rectangle [id:dp661407400810794] 
		\draw  [fill={rgb, 255:red, 146; green, 108; blue, 75 }  ,fill opacity=1 ] (137.34,380.7) -- (369.11,380.7) -- (369.11,419.99) -- (137.34,419.99) -- cycle ;
		%Rounded Rect [id:dp14042717508972213] 
		\path  [shading=_l1th8ezcg,_audied9e7] (356.84,65.35) .. controls (356.84,64.05) and (357.89,63) .. (359.19,63) -- (361.15,63) .. controls (362.45,63) and (363.5,64.05) .. (363.5,65.35) -- (363.5,369.65) .. controls (363.5,370.95) and (362.45,372) .. (361.15,372) -- (359.19,372) .. controls (357.89,372) and (356.84,370.95) .. (356.84,369.65) -- cycle ; % for fading 
		\draw   (356.84,65.35) .. controls (356.84,64.05) and (357.89,63) .. (359.19,63) -- (361.15,63) .. controls (362.45,63) and (363.5,64.05) .. (363.5,65.35) -- (363.5,369.65) .. controls (363.5,370.95) and (362.45,372) .. (361.15,372) -- (359.19,372) .. controls (357.89,372) and (356.84,370.95) .. (356.84,369.65) -- cycle ; % for border 

		%Shape: Spring [id:dp16658294194788303] 
		\draw  [color={rgb, 255:red, 74; green, 74; blue, 74 }  ,draw opacity=1 ][line width=0.75]  (436.59,100.06) .. controls (440.27,100.55) and (443.96,102.52) .. (443.96,106.44) .. controls (443.96,114.3) and (429.22,114.3) .. (429.22,110.37) .. controls (429.22,106.44) and (443.96,106.44) .. (443.96,114.3) .. controls (443.96,122.16) and (429.22,122.16) .. (429.22,118.23) .. controls (429.22,114.3) and (443.96,114.3) .. (443.96,122.16) .. controls (443.96,130.01) and (429.22,130.01) .. (429.22,126.09) .. controls (429.22,122.16) and (443.96,122.16) .. (443.96,130.01) .. controls (443.96,137.87) and (429.22,137.87) .. (429.22,133.94) .. controls (429.22,130.01) and (443.96,130.01) .. (443.96,137.87) .. controls (443.96,145.73) and (429.22,145.73) .. (429.22,141.8) .. controls (429.22,137.87) and (443.96,137.87) .. (443.96,145.73) .. controls (443.96,153.58) and (429.22,153.58) .. (429.22,149.66) .. controls (429.22,145.73) and (443.96,145.73) .. (443.96,153.58) .. controls (443.96,161.44) and (429.22,161.44) .. (429.22,157.51) .. controls (429.22,153.58) and (443.96,153.58) .. (443.96,161.44) .. controls (443.96,169.3) and (429.22,169.3) .. (429.22,165.37) .. controls (429.22,161.44) and (443.96,161.44) .. (443.96,169.3) .. controls (443.96,177.15) and (429.22,177.15) .. (429.22,173.23) .. controls (429.22,169.3) and (443.96,169.3) .. (443.96,177.15) .. controls (443.96,185.01) and (429.22,185.01) .. (429.22,181.08) .. controls (429.22,177.15) and (443.96,177.15) .. (443.96,185.01) .. controls (443.96,192.87) and (429.22,192.87) .. (429.22,188.94) .. controls (429.22,185.01) and (443.96,185.01) .. (443.96,192.87) .. controls (443.96,200.72) and (429.22,200.72) .. (429.22,196.8) .. controls (429.22,192.87) and (443.96,192.87) .. (443.96,200.72) .. controls (443.96,208.58) and (429.22,208.58) .. (429.22,204.65) .. controls (429.22,200.72) and (443.96,200.72) .. (443.96,208.58) .. controls (443.96,212.51) and (440.27,214.47) .. (436.59,214.96) ;
		%Shape: Rectangle [id:dp22739131790719957] 
		\path  [shading=_7fov45wgl,_zw799en1h] (340.72,84.94) -- (346.77,84.94) -- (346.77,86.73) -- (340.72,86.73) -- cycle ; % for fading 
		\draw  [color={rgb, 255:red, 0; green, 0; blue, 0 }  ,draw opacity=1 ] (340.72,84.94) -- (346.77,84.94) -- (346.77,86.73) -- (340.72,86.73) -- cycle ; % for border 

		%Rounded Rect [id:dp8164684948783012] 
		\path  [shading=_l971no8s8,_ma7ryb15k] (365.18,84.51) .. controls (365.18,84.02) and (365.58,83.62) .. (366.07,83.62) -- (456.62,83.62) .. controls (457.1,83.62) and (457.5,84.02) .. (457.5,84.51) -- (457.5,87.16) .. controls (457.5,87.65) and (457.1,88.04) .. (456.62,88.04) -- (366.07,88.04) .. controls (365.58,88.04) and (365.18,87.65) .. (365.18,87.16) -- cycle ; % for fading 
		\draw   (365.18,84.51) .. controls (365.18,84.02) and (365.58,83.62) .. (366.07,83.62) -- (456.62,83.62) .. controls (457.1,83.62) and (457.5,84.02) .. (457.5,84.51) -- (457.5,87.16) .. controls (457.5,87.65) and (457.1,88.04) .. (456.62,88.04) -- (366.07,88.04) .. controls (365.58,88.04) and (365.18,87.65) .. (365.18,87.16) -- cycle ; % for border 

		%Rounded Same Side Corner Rect [id:dp3337767177690609] 
		\draw  [fill={rgb, 255:red, 236; green, 148; blue, 2 }  ,fill opacity=1 ] (423.11,238.8) .. controls (423.11,233.23) and (427.62,228.71) .. (433.19,228.71) -- (441.99,228.71) .. controls (447.56,228.71) and (452.08,233.23) .. (452.08,238.8) -- (452.08,259.09) .. controls (452.08,259.09) and (452.08,259.09) .. (452.08,259.09) -- (423.11,259.09) .. controls (423.11,259.09) and (423.11,259.09) .. (423.11,259.09) -- cycle ;
		%Straight Lines [id:da9036000547278182] 
		\draw [color={rgb, 255:red, 74; green, 74; blue, 74 }  ,draw opacity=1 ]   (437.57,228.22) -- (437.57,213.98) ;
		%Rounded Rect [id:dp8364710468830279] 
		\path  [shading=_u0fs4tjoa,_f6tcg567o] (347.44,81.14) .. controls (347.44,79.59) and (348.69,78.34) .. (350.24,78.34) -- (372.21,78.34) .. controls (373.75,78.34) and (375.01,79.59) .. (375.01,81.14) -- (375.01,89.53) .. controls (375.01,91.07) and (373.75,92.32) .. (372.21,92.32) -- (350.24,92.32) .. controls (348.69,92.32) and (347.44,91.07) .. (347.44,89.53) -- cycle ; % for fading 
		\draw  [color={rgb, 255:red, 74; green, 74; blue, 74 }  ,draw opacity=1 ] (347.44,81.14) .. controls (347.44,79.59) and (348.69,78.34) .. (350.24,78.34) -- (372.21,78.34) .. controls (373.75,78.34) and (375.01,79.59) .. (375.01,81.14) -- (375.01,89.53) .. controls (375.01,91.07) and (373.75,92.32) .. (372.21,92.32) -- (350.24,92.32) .. controls (348.69,92.32) and (347.44,91.07) .. (347.44,89.53) -- cycle ; % for border 

		%Shape: Ellipse [id:dp5179670735428143] 
		\draw  [fill={rgb, 255:red, 0; green, 0; blue, 0 }  ,fill opacity=1 ] (331.3,84.94) .. controls (331.3,80.52) and (333.41,76.93) .. (336.01,76.93) .. controls (338.61,76.93) and (340.72,80.52) .. (340.72,84.94) .. controls (340.72,89.37) and (338.61,92.95) .. (336.01,92.95) .. controls (333.41,92.95) and (331.3,89.37) .. (331.3,84.94) -- cycle ;
		%Shape: Rectangle [id:dp7277974618781018] 
		\path  [shading=_ifkxbtcrd,_kvu3t6wn8] (339.24,125.7) -- (345.29,125.7) -- (345.29,127.49) -- (339.24,127.49) -- cycle ; % for fading 
		\draw  [color={rgb, 255:red, 0; green, 0; blue, 0 }  ,draw opacity=1 ] (339.24,125.7) -- (345.29,125.7) -- (345.29,127.49) -- (339.24,127.49) -- cycle ; % for border 

		%Rounded Rect [id:dp9330549543329489] 
		\path  [shading=_i9ec3u4oc,_relekxdrh] (345.97,121.89) .. controls (345.97,120.35) and (347.22,119.1) .. (348.76,119.1) -- (370.74,119.1) .. controls (372.28,119.1) and (373.53,120.35) .. (373.53,121.89) -- (373.53,130.28) .. controls (373.53,131.83) and (372.28,133.08) .. (370.74,133.08) -- (348.76,133.08) .. controls (347.22,133.08) and (345.97,131.83) .. (345.97,130.28) -- cycle ; % for fading 
		\draw  [color={rgb, 255:red, 74; green, 74; blue, 74 }  ,draw opacity=1 ] (345.97,121.89) .. controls (345.97,120.35) and (347.22,119.1) .. (348.76,119.1) -- (370.74,119.1) .. controls (372.28,119.1) and (373.53,120.35) .. (373.53,121.89) -- (373.53,130.28) .. controls (373.53,131.83) and (372.28,133.08) .. (370.74,133.08) -- (348.76,133.08) .. controls (347.22,133.08) and (345.97,131.83) .. (345.97,130.28) -- cycle ; % for border 

		%Shape: Ellipse [id:dp27429498112957074] 
		\draw  [fill={rgb, 255:red, 0; green, 0; blue, 0 }  ,fill opacity=1 ] (329.83,125.7) .. controls (329.83,121.28) and (331.94,117.69) .. (334.54,117.69) .. controls (337.14,117.69) and (339.24,121.28) .. (339.24,125.7) .. controls (339.24,130.12) and (337.14,133.71) .. (334.54,133.71) .. controls (331.94,133.71) and (329.83,130.12) .. (329.83,125.7) -- cycle ;
		%Straight Lines [id:da6364752459654412] 
		\draw [color={rgb, 255:red, 110; green, 61; blue, 18 }  ,draw opacity=1 ][line width=0.75]    (373.53,125.85) -- (380.9,125.85) ;
		%Shape: Rectangle [id:dp879583802615739] 
		\draw  [color={rgb, 255:red, 121; green, 58; blue, 3 }  ,draw opacity=1 ][fill={rgb, 255:red, 97; green, 47; blue, 4 }  ,fill opacity=1 ] (380.41,122.91) -- (380.9,122.91) -- (380.9,127.82) -- (380.41,127.82) -- cycle ;
		%Shape: Rectangle [id:dp2590872105368831] 
		\draw  [color={rgb, 255:red, 0; green, 0; blue, 0 }  ,draw opacity=1 ][fill={rgb, 255:red, 238; green, 188; blue, 105 }  ,fill opacity=1 ][line width=0.75]  (402.49,420.97) -- (381.86,420.97) -- (381.86,88.53) -- (402.49,88.53) -- cycle ;
		%Straight Lines [id:da4342154910571179] 
		\draw    (402.49,408.51) -- (389.72,408.51) ;
		%Straight Lines [id:da8960085846992045] 
		\draw    (402.49,398.04) -- (389.72,398.04) ;
		%Straight Lines [id:da29417082715176046] 
		\draw    (402.49,388.57) -- (389.72,388.57) ;
		%Straight Lines [id:da7043382052247802] 
		\draw    (402.49,378.6) -- (389.72,378.6) ;
		%Straight Lines [id:da14461348672257412] 
		\draw    (402.49,368.64) -- (389.72,368.64) ;
		%Straight Lines [id:da8812764216308848] 
		\draw    (402.49,358.67) -- (389.72,358.67) ;
		%Straight Lines [id:da46113320559838833] 
		\draw    (402.49,348.7) -- (389.72,348.7) ;
		%Straight Lines [id:da06870609801527894] 
		\draw    (402.49,338.73) -- (389.72,338.73) ;
		%Straight Lines [id:da6491635747766802] 
		\draw    (402.49,328.76) -- (389.72,328.76) ;
		%Straight Lines [id:da6541158039132522] 
		\draw    (402.49,318.8) -- (389.72,318.8) ;
		%Straight Lines [id:da8229925427369018] 
		\draw    (402.49,308.83) -- (389.72,308.83) ;
		%Straight Lines [id:da9289832823167388] 
		\draw    (402.49,298.86) -- (389.72,298.86) ;
		%Straight Lines [id:da1480662946592668] 
		\draw    (402.49,288.89) -- (389.72,288.89) ;
		%Straight Lines [id:da8076788219370534] 
		\draw    (402.49,280.92) -- (389.72,280.92) ;
		%Straight Lines [id:da7862953107859727] 
		\draw    (402.49,270.95) -- (389.72,270.95) ;
		%Straight Lines [id:da3509670935278919] 
		\draw    (402.49,260.98) -- (389.72,260.98) ;
		%Straight Lines [id:da9062043984578614] 
		\draw    (402.49,251.01) -- (389.72,251.01) ;
		%Straight Lines [id:da636865271801506] 
		\draw    (402.49,241.04) -- (389.72,241.04) ;
		%Straight Lines [id:da883281904699244] 
		\draw    (402.49,231.08) -- (389.72,231.08) ;
		%Straight Lines [id:da2697651405327248] 
		\draw    (402.49,221.11) -- (389.72,221.11) ;
		%Straight Lines [id:da01060565079167164] 
		\draw    (402.49,211.14) -- (389.72,211.14) ;
		%Straight Lines [id:da049150025088934735] 
		\draw    (402.49,201.17) -- (389.72,201.17) ;
		%Straight Lines [id:da7848421068801716] 
		\draw    (402.49,191.2) -- (389.72,191.2) ;
		%Straight Lines [id:da6766645206515829] 
		\draw    (402.49,181.24) -- (389.72,181.24) ;
		%Straight Lines [id:da3941559797200511] 
		\draw    (402.49,171.27) -- (389.72,171.27) ;
		%Straight Lines [id:da6345888877595267] 
		\draw    (402.49,161.3) -- (389.72,161.3) ;
		%Straight Lines [id:da5187847923306577] 
		\draw    (402.49,151.33) -- (389.72,151.33) ;
		%Straight Lines [id:da9569200064358407] 
		\draw    (402.49,141.36) -- (389.72,141.36) ;
		%Straight Lines [id:da6246783306251273] 
		\draw    (402.49,131.4) -- (389.72,131.4) ;
		%Straight Lines [id:da3271985919064788] 
		\draw    (402.49,121.43) -- (389.72,121.43) ;
		%Straight Lines [id:da0483997763552233] 
		\draw    (402.49,111.46) -- (389.72,111.46) ;
		%Straight Lines [id:da15577476855891215] 
		\draw    (402.49,101.49) -- (389.72,101.49) ;

		%Shape: Rectangle [id:dp17328219705210146] 
		\draw  [color={rgb, 255:red, 121; green, 58; blue, 3 }  ,draw opacity=1 ][fill={rgb, 255:red, 97; green, 47; blue, 4 }  ,fill opacity=1 ] (402.96,122.91) -- (403.45,122.91) -- (403.45,127.82) -- (402.96,127.82) -- cycle ;
		%Shape: Rectangle [id:dp9738368237348072] 
		\draw  [fill={rgb, 255:red, 74; green, 74; blue, 74 }  ,fill opacity=1 ] (258.63,367) -- (369.11,367) -- (369.11,380.7) -- (258.63,380.7) -- cycle ;
		%Straight Lines [id:da8313423755328997] 
		\draw [color={rgb, 255:red, 74; green, 74; blue, 74 }  ,draw opacity=1 ]   (437,100.25) -- (437,95.25) ;
		%Shape: Arc [id:dp6175334292069026] 
		\draw  [draw opacity=0] (438.93,83.06) .. controls (438.38,82.39) and (437.72,82) .. (437,82) .. controls (435.07,82) and (433.5,84.85) .. (433.5,88.38) .. controls (433.5,91.9) and (435.07,94.75) .. (437,94.75) .. controls (438.9,94.75) and (440.45,91.98) .. (440.5,88.53) -- (437,88.38) -- cycle ; \draw  [color={rgb, 255:red, 74; green, 74; blue, 74 }  ,draw opacity=1 ] (438.93,83.06) .. controls (438.38,82.39) and (437.72,82) .. (437,82) .. controls (435.07,82) and (433.5,84.85) .. (433.5,88.38) .. controls (433.5,91.9) and (435.07,94.75) .. (437,94.75) .. controls (438.9,94.75) and (440.45,91.98) .. (440.5,88.53) ;

		% Text Node
		\draw (230,393) node [anchor=north west][inner sep=0.75pt]   [align=left] {{\fontfamily{pcr}\selectfont \textcolor[rgb]{1,0.92,0}{TABLE}}};
		% Text Node
		\draw (270,241) node [anchor=north west][inner sep=0.75pt]   [align=left] {{\fontfamily{pcr}\selectfont Ring Stand}};
		% Text Node
		\draw (235,116) node [anchor=north west][inner sep=0.75pt]   [align=left] {{\fontfamily{pcr}\selectfont Ruler Clamp}};
		% Text Node
		\draw (374.9,63.5) node [anchor=north west][inner sep=0.75pt]   [align=left] {{\fontfamily{pcr}\selectfont Spring Clamp}};
		% Text Node
		\draw (450,151.5) node [anchor=north west][inner sep=0.75pt]   [align=left] {{\fontfamily{pcr}\selectfont Spring}};
		% Text Node
		\draw (407.5,303) node [anchor=north west][inner sep=0.75pt]   [align=left] {{\fontfamily{pcr}\selectfont Ruler}};
		% Text Node
		\draw (457.25,237.5) node [anchor=north west][inner sep=0.75pt]   [align=left] {{\fontfamily{pcr}\selectfont Known Mass}};
		% Text Node
		\draw (425,240) node [anchor=north west][inner sep=0.75pt]   [align=left] {{\fontfamily{pcr}\selectfont {\scriptsize 500g}}};
	\end{tikzpicture}
	% \captionsetup{singlelinecheck = false, justification=justified}
	\caption{Setup Diagram\label{figure1}}
\end{figure}


The figure above displays the setup used for the activity.
To calculate displacement, first the length of the springs at equilibrium were measured and noted.
Next, when masses were added to the springs, the length of the springs were measured with the ruler where only the length of the coil was measured and not including the rings.
Lastly, to calculate displacement, the spring length at equilibrium was subtracted from the stretched spring length.
\section{Observations}
\begin{table}[htpb]
	\centering
	\caption{Force and Displacement Data for Two Springs\label{table1}}

	\begin{tabular*}{0.9\textwidth}{l@{\extracolsep{\fill}}ccc}
		\toprule
		\textbf{Mass (kg)} &
		\textbf{Force (N)} &
		\textbf{\begin{tabular}[c]{@{}c@{}}Spring A\\ Displacement (cm)\end{tabular}} &
		\textbf{\begin{tabular}[c]{@{}c@{}}Spring B\\ Displacement (cm)\end{tabular}} \\ \midrule
		0.20 & 1.96 & 2.20 & 0.270 \\
		0.40 & 3.92 & 10.3 & 0.450 \\
		0.60 & 5.67 & 19.1 & 0.920 \\
		0.80 & 7.85 & 28.1 & 1.43  \\
		1.0  & 9.81 & 37.6 & 1.91  \\
		1.2  & 11.8 & 45.1 & 2.40  \\ \bottomrule
	\end{tabular*}
\end{table}

\newpage

\section{Analysis}
\vspace{-4mm}
\begin{figure}[H]
	\caption{Force Vs. Displacement for Spring One\label{figure1}}
	\vspace{-4mm}
	\begin{tikzpicture}
		\begin{axis}[
				xmin=0, xmax=40,
				ymin=0, ymax=12,
				xtick distance = 5,
				ytick distance = 2,
				xticklabel style={
						/pgf/number format/precision=3,
						/pgf/number format/fixed},
				grid = both,
				minor tick num = 1,
				major grid style = {lightgray},
				minor grid style = {lightgray!25},
				width = \textwidth,
				height = 0.5\textwidth,
				xlabel = {Displacement (cm)},
				ylabel = {Force (N)},
			]

			\addplot[red, only marks] file[skip first] {./Data/spring1.dat};
			\addplot [color=blue, domain=0:40, mark=none smooth, thick] {0.235*x+1.4};

		\end{axis}
	\end{tikzpicture}
\end{figure}
\vspace{-8mm}
\hspace{8mm}
\textit{$\star$ Blue line represents Spring A and green line represents Spring B}
\begin{figure}[H]
	\caption{Force Vs. Displacement for Spring Two\label{figure2}}
	\vspace{-4mm}
	\begin{tikzpicture}
		\begin{axis}[
				xmin=0, xmax=5,
				ymin=0, ymax=25,
				xtick distance = 1,
				ytick distance = 5,
				xticklabel style={
						/pgf/number format/precision=3,
						/pgf/number format/fixed},
				grid = both,
				minor tick num = 1,
				major grid style = {lightgray},
				minor grid style = {lightgray!25},
				width = \textwidth,
				height = 0.5\textwidth,
				xlabel = {Displacement (cm)},
				ylabel = {Force (N)},
			]

			\addplot[red, only marks] file[skip first] {./Data/spring2.dat};
			\addplot [color=blue, domain=0:5, mark=none smooth, thick] {3.86*x+2.21};

		\end{axis}
	\end{tikzpicture}
\end{figure}
\vspace{-8mm}
\hspace{1mm}
\textit{$\star$ Blue line represents Spring B and green line represents Spring A}
\begin{enumerate}[font=\bfseries]
	\item \textbf{Calculating Slope of Each Graph:}
	      \vspace{-4mm}
	      \begin{enumerate}
		      \item \underline{Spring A}\\
		            The trendline's equation for Spring A is:~\bm{$y = 0.225x + 1.48$}\\
		            $\therefore$ The slope of the trendline is 0.225.
		      \item \underline{Spring B}\\
		            The trendline's equation for Spring B is:~\bm{$y = 4.38x + 1.44$}\\
		            $\therefore$ The slope of the trendline is 4.38.
	      \end{enumerate}
	      \newpage
	\item \textbf{Relationship Between Force and Displacement:}\\
	      There is a linear relationship between force and displacement, where force is directly proportional to displacement ($F \propto \Delta d$).
	\item \textbf{Physical Quantity the Slope Represents:}\\
	      The slope represents the spring constant, which physically determines the stiffness of the spring.
	      The two springs have different slope values because each spring has a different spring constant, where a higher slope value represents a stiffer spring.
	      This holds true because Spring B has a larger slope than Spring A, meaning it should be more stiff, evident through Spring B's smaller displacement values than Spring A.
	\item \textbf{Equation of Line from Proportionality Statement:}\\
	      Let $k$ represent the slope of the line.\\
	      Let $x$ represent the displacement ($\Delta d$) of the spring from equilibrium.
	      \[
		      \begin{aligned}
			      F      & \propto x \\
			      \bm{F} & = \bm{kx}
		      \end{aligned}
	      \]
	\item \textbf{Properties of an "Ideal Spring":}\\
	      The properties of an ideal spring are that it is frictionless and massless.
	      Additionally, an ideal spring should obey Hooke's Law which states that the force exerted by the spring is proportional to the displacement of the spring from its relaxed position.
	      The springs used in class for the activity are not examples of ideal springs as they have mass and friction to some degree.
	      Also, the springs do not perfectly obey Hooke's Law as the data points are not perfectly linear or have the y-intercept of the trendline at $y=0$.
	      Ideal springs exist in theory to simplify calculations while providing acceptable accuracy for practical problems.
\end{enumerate}

\end{document}