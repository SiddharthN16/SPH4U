\documentclass[12pt,letterpaper]{article}
\newcommand\tab[1][1cm]{\hspace*{#1}}
\usepackage[utf8]{inputenc}
\usepackage[letterpaper]{geometry}
\usepackage{amsmath}
\usepackage{gensymb}
\usepackage{amssymb}
\usepackage{rotating}
\usepackage{fancyhdr}
\usepackage{setspace}
\usepackage{mathtools}
\usepackage{float}
\usepackage{caption}
\usepackage{bm}
\usepackage{pgfplots}
\usepackage{pgfplotstable}
\pgfplotsset{compat=newest}

\captionsetup[table]{skip=0pt,singlelinecheck=off}
\doublespacing 

\geometry{top=1.0in, bottom=1.0in, left=1.0in, right=1.0in}
\renewcommand{\headrulewidth}{0pt} 
\renewcommand{\footrulewidth}{0pt} 
\setlength\headsep{0.333in}


\title{\textbf{Galileo's Ramp Lab}}
\author{Siddharth Nema \\Partner: Andy Zhang}

\begin{document}
\maketitle
\newpage
\section{Observations}

\begin{table}[H]
  \caption{Height and Angle Data\label{table1}}
  \begin{tabular*}{\textwidth}{l@{\extracolsep{\fill}}cccc}
    \hline
    \textbf{Trial Number} & \textbf{Track Height (m)} & \textbf{Average Time (s)} & \textbf{Angle of Incline (\textdegree)}\\
    \hline
    \textbf{Height \#1}& 0.13m& 2.56s& 5.07\textdegree\\
    \textbf{Height \#2}& 0.20m& 1.98s& 7.66\textdegree\\
    \textbf{Height \#3}& 0.27m& 1.66s& 10.07\textdegree\\
    \textbf{Height \#4}& 0.34m& 1.38s& 12.98\textdegree\\
    \textbf{Height \#5}& 0.41m& 1.30s& 15.68\textdegree\\
    \hline
  \end{tabular*}
\end{table}
\vspace{-4mm}
\textbf{Length of Ramp (\bm{$\Delta d$})} = 1.50 metres

\section{Analysis}

\textbf{1. Calculating Acceleration of the Ball:} \\
\underline{Sample Calculation for Height \#1:}\\
\vspace{-12mm}
\begin{center}
  \bm{$\vec{a}=\frac{2\Delta d}{t^2}$}\\
  \bm{$\vec{a}=\frac{2(1.50)}{2.56^2}$}\\
  \bm{$\vec{a}=0.46$} \textbf{m/s\textsuperscript{2}[downhill]}

\end{center}

\begin{table}[H]
  \caption{Angles and Acceleration Data\label{table2}}
  \begin{tabular*}{\textwidth}{l@{\extracolsep{\fill}}cccc}
    \hline
    \textbf{Height (m)} & \textbf{Angle (\textdegree)} & \textbf{Sin \bm{$\theta$}} & \textbf{Acceleration (m/s\textsuperscript{2})} \\
    \hline
    \textbf{Height \#1:} 0.13m& 5.07\textdegree& 0.09& 0.46m/s\textsuperscript{2}\\
    \textbf{Height \#2:} 0.20m& 7.66\textdegree& 0.13& 0.77m/s\textsuperscript{2}\\
    \textbf{Height \#3:} 0.27m& 10.07\textdegree& 0.18& 1.09m/s\textsuperscript{2}\\
    \textbf{Height \#4:} 0.34m& 12.98\textdegree& 0.23& 1.58m/s\textsuperscript{2}\\
    \textbf{Height \#5:} 0.41m& 15.68\textdegree& 0.27& 1.78m/s\textsuperscript{2}\\
    \hline
  \end{tabular*}
\end{table}
\newpage

\noindent
\textbf{2. Acceleration Vs. Sin \bm{$\theta$} Graph Without Trendline:}

\begin{figure}[H]
  \caption{Acceleration Vs. Sin $\theta$\label{figure1}}
  \vspace{-4mm}
  \begin{tikzpicture}
    \begin{axis}[
        xmin=0, xmax=0.3,
        ymin=0, ymax=2,
        xtick distance = 0.05,
        ytick distance = 0.2,
        xticklabel style={
            /pgf/number format/precision=3,
            /pgf/number format/fixed},
        grid = both,
        minor tick num = 1,
        major grid style = {lightgray},
        minor grid style = {lightgray!25},
        width = \textwidth,
        height = 0.5\textwidth,
        xlabel = {Sin $\theta$},
        ylabel = {Acceleration (m/s\textsuperscript{2})},
      ]

      \addplot[red, only marks] file[skip first] {result.dat};

    \end{axis}
  \end{tikzpicture}
\end{figure}

\noindent
\textbf{3. Acceleration Vs. Sin \bm{$\theta$} Graph With Trendline:}
\begin{figure}[H]
  \caption{Acceleration Vs. Sin $\theta$\label{figure2}}
  \vspace{-4mm}
  \begin{tikzpicture}
    \begin{axis}[
        xmin=0, xmax=0.3,
        ymin=0, ymax=2,
        xtick distance = 0.05,
        ytick distance = 0.2,
        xticklabel style={
            /pgf/number format/precision=3,
            /pgf/number format/fixed},
        grid = both,
        minor tick num = 1,
        major grid style = {lightgray},
        minor grid style = {lightgray!25},
        width = \textwidth,
        height = 0.5\textwidth,
        xlabel = {Sin $\theta$},
        ylabel = {Acceleration (m/s\textsuperscript{2})},
      ]

      \addplot[red, only marks] file[skip first] {result.dat};
      \addplot [color=blue, domain=0.088:0.3, mark=none smooth, thick] {7.5727*x-0.2135};

    \end{axis}
  \end{tikzpicture}
\end{figure}
\vspace{-4mm}

\noindent
\textbf{\tab a) Relation between Acceleration and Sin \bm{$\theta$}:}

\noindent
\tab There is a linear relationship between acceleration and sin $\theta$, which represents the \tab acceleration due to gravity.

\newpage

\noindent
\textbf{\tab b) Calculating Slope of Trendline:}

\begin{center}
  \bm{$m = \frac{y_{2}-y_{1}}{x_{2}-x_{1}}$}\\
  \bm{$m=\frac{1.09-0.46}{0.175-0.088}$}\\
  \bm{$m=7.24$}
\end{center}
\vspace{-4mm}
\noindent
\tab $\therefore$ The slope of the trendline is 7.24.

\vspace{4mm}
\noindent
\textbf{\tab c) Equation of the Line:}

\noindent
\tab Equation based on Excel trendline: \bm{$y=7.5727x - 0.2135$} \textsuperscript{$\star$}

\noindent
\tab Equation from calculations: \bm{$y = 7.24x-0.177$} \textsuperscript{$\star$} \\

\noindent
\tab \textit{$\star$ Where $x$ truly represents $\sin x$}

\vspace{4mm}
\noindent
\textbf{\tab d) Acceleration of the Ball if \bm{$\theta=90$}\textdegree:}

\noindent
\tab Calculated using equation from calculations:
\begin{center}
  \bm{$y=7.24 \sin x-0.177$}\\
  \bm{$y=7.24 \sin 90\degree -0.177$}\\
  \bm{$y=7.24 - 0.177$}\\
  \bm{$y=7.06$}
\end{center}
\vspace{-4mm}
\noindent
\tab $\therefore$ The calculated acceleration of the ball at 90\textdegree\ would be 7.06 m/s\textsuperscript{2}.

\vspace{4mm}
\noindent
\textbf{\tab e) Calculating the Percentage Error:}

\begin{center}
  \bm{$\delta = \left\lvert \frac{ \mathcal{V}_{A} - \mathcal{V}_{E} }{\mathcal{V}_{E}} \right\rvert \centerdot 100\%$}\\
  \bm{$\delta = \left\lvert \frac{9.8-7.06}{7.06}\right\rvert \centerdot 100\% $} \\
  \bm{$\delta = 39\%$}
\end{center}
\vspace{-4mm}
\noindent
\tab $\therefore$ The calculated percentage error is 39\%.

\newpage

\noindent
\textbf{\tab f) Discussing Two (2) Sources of Error:}

\noindent
\tab \textbf{1)} Since the ball is travelling on a surface, there are other forces acting on it, which \tab slow down the ball’s motion such as friction between the ball and surface as well as \tab air resistance. These two forces oppose the ball’s motion, preventing it from reaching \tab the target 9.8 m/s\textsuperscript{2} acceleration. The material of the bouncy ball was rubber and was \tab against semi rough plastic, which would result in a high force of friction that would \tab affect the ball as “rolling friction” which we cannot account for currently.

\vspace{4mm}

\noindent
\tab \textbf{2)} In addition, due to the poor design of the ramp we used, the rubber ball kept \tab hitting the sides of the ramp as it rolled down, which would decrease its speed and \tab have external forces act upon it. This would increase in time and therefore, introduce \tab a higher \% error in our calculations.

\vspace{8mm}
\noindent
\textbf{4. After performing his experiments using inclined ramps and balls of different masses, Galileo concluded that the acceleration of various balls down a ramp was independent of mass.}

\noindent
\tab \textbf{a) Modifying The Experiment to Test Galileo's Conclusion:}

\noindent
\tab The experiment could be modified by using multiple balls of varying masses with the \tab same five angles and creating acceleration vs sin $\theta$ graphs for all of them. The graphs \tab should roughly all look the same or only have slight differences due to experimental \tab errors.

\noindent
\tab \textbf{b) Explaining Why the Feather and Bowling Ball Result does not Contradict \tab Galileo's Result:}

\noindent
\tab The result does not contradict Galileo’s conclusion because when the objects are \tab dropped, other forces act on them; primarily air resistance, which is not independent \tab of mass. As a result the feather experiences more air resistance than the bowling ball. \tab If the two objects were placed in a complete vacuum, both objects would reach the \tab ground at the same time, as gravity is the only force acting on the objects, displaying \tab that acceleration is independent of mass.

\end{document}