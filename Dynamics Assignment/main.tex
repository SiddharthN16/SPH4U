\documentclass{docs}

\title{SPH4U: Dynamics Assignment}

\begin{document}
\thispagestyle{firstpage}
\textbf{\thetitle}

\begin{prob}
  Two heavy boxes, $m_{1}$ and $m_{2}$, lie stationary on different inclines, as shown.
  A rope runs over a pulley and connects the boxes.
  Mass 1, $m_{1}$ is 380kg. Assuming that each incline is frictionless and the system is in equilibrium, answer the
  following:

  \begin{enumerate}[(a)]
    \item \textbf{Draw FBD(s).}\\
          \textit{Insert Free Body Diagram Here}
    \item \textbf{Find the magnitude of the tension in the cable.}\\
          \begin{sol}
            \vec{F_{net_{1x}}}            & = 0                                                              \\
            0                             & = \vec{F_{g_{1}}} \sin 18\degree + \left\lvert F_{T}\right\rvert \\
            0                             & = m_{1}g \sin 18 \degree + \left\lvert F_{T}\right\rvert         \\
            0                             & = 380(-9.8) \sin 18 \degree + \left\lvert F_{T}\right\rvert      \\
            \left\lvert F_{T}\right\rvert & = 1150.78\text{N}                                                \\
            \left\lvert F_{T}\right\rvert & = 1200 \text{N}                                                  \\
          \end{sol}
          $\therefore$ The magnitude of tension in the string is 1200N.

    \item \textbf{Calculate the mass of m2 needed to keep the system in equilibrium.}

          \begin{sol}
            F_{net_{2x}} & = 0                                         \\
            0            & = \vec{F_{g_{2}}} \sin 33 \degree + 1150.78 \\
            m_{2}        & = \frac{-1150.78}{-9.8\sin 33 \degree}      \\
            m_{2}        & = 215.6\text{kg}                            \\
            m_{2}        & = 220 \text{kg}
          \end{sol}
          $\therefore$ The mass of the second box ($m_{2}$) is 220kg.
  \end{enumerate}
\end{prob}

\begin{prob}
  A girl applies a 140N force to a 35.0kg bale of hay at an angle of 28\textdegree above the horizontal. The force of friction acting on the bale is 55N.
  \begin{enumerate}[(a)]
    \item \textbf{What will be the horizontal acceleration of the bale?}
          \begin{sol}
            \vec{F_{net_{x}}} & = \vec{F_{a_{x}}} - \vec{F_{f}} & \vec{F_{net_{x}}} & = m\vec{a}                                 \\
            & = 140 \cos 28\degree - 55       &\tab 66.81             & = 35\vec{a}                                \\
            & = 66.81 \text{N[right]}         &\tab \vec{a}           & = 1.96\text{m/s\textsuperscript{2}[right]} \\
            &                                 &\tab \vec{a}           & = 2.0\text{m/s\textsuperscript{2}[right]}
          \end{sol}
          $\therefore$ The horizontal acceleration is 2.0m/s\textsuperscript{2}[right]
          \newpage
    \item \textbf{What is the cofficient of friction between the bale and the ground?}
          \begin{sol}
            \vec{F_{f}} &= \mu_{f}\vec{F_{N}}\\
            55 &= \mu_{f}(35\times 9.8)\\
            \mu_{f} &= \frac{55}{35\times 9.8}\\
            \mu_{f} &= 0.16
          \end{sol}
  \end{enumerate}
  $\therefore$ The coefficient of friction between the bale and ground is 0.16.
\end{prob}

\begin{prob}
  A 15kg box is pushed upa 35\textdegree ramp. A force of 110N exists between the box adn the ramp.
  \begin{enumerate}[(a)]
    \item \textbf{Draw an FBD showing a tilted coordinate system (label positive x-direction)}
          \textit{Draw FBD Diagram}
    \item \textbf{What minimum force, \bm{$F$}, would be  necessary to move the box up the ramp at a constant speed?}
          \begin{sol}
            \vec{F_{net_{x}}} &= \vec{F_{a}} + \vec{F_{g_{x}}} + \vec{F_{f}}\\
            0 &= \vec{F_{a}} + 15(-9.8)\sin 35\degree - 110\\
            0 &= \vec{F_{a}} - 194.31\\
            \vec{F_{a}} &= 194.31\text{N}
          \end{sol}
          $\therefore$ Since 194.31N is the force required to keep the object at rest, the minimum force for constant speed must be great than 194.31N.
  \end{enumerate}
\end{prob}
\end{document}